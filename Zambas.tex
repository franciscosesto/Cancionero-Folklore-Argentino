
 
\beginsong{Agitando Pañuelos}[by={Adolfo Ábalos}]

		\phantomsection  \addcontentsline{toc}{section}{Agitando Pañuelos} 
 \label{sec:Agitando_Pañuelos} \beginverse
		\[Em]Te \[E7]vi, no olvida\[Am]ré un carna\[D7]val, guitarra, bombo y vio\[G]lín \[B7] \\
		\[Em]Agitando pa\[D]ñuelos te \[C]vi, cadencia al \[B7]bailar, airosa el \[Am]perfil \[G] \[B7] \[Em] \\
		\endverse
		
		\beginverse
		\[E7]Me \[Am]fui diciendo a\[D7]diós y en ese a\[G]diós quedó enredado un \[B7]querer \\
		\[Em]Agitando pa\[D]ñuelos me \[C]fui ¡Qué lindo \[B7]añorar tu zamba de \[Am]ayer! \[G] \[B7] \[Em] \\
		\endverse
		
		\beginchorus
		\[Em]Yo me \[E7]iré tu ven\[Am]drá Yo te lle\[D7]varé, mi rancho se alegre\[G]rá \[B7] \\
		\[Em]Agitando pa\[D]ñuelos me \[C]iré y en mí vivi\[B7]rá aquel carna\[Am]val \[G] \[B7] \[Em] \\
		\[Em]Agitando pa\[D]ñuelos me \[C]iré cantando esta \[B7]zamba repiquetea\[Am]dita \[G] \[B7] \[Em]
		\endchorus
		
		\beginverse
		\[Em]Vol\[E7]ví y te encon\[Am]tré... toda mi \[D7]voz le dio a la copla un can\[G]tar \[B7] \\
		\[Em]Agitando pa\[D]ñuelos vol\[C]ví, sintiendo \[B7]también mi pecho agi\[Am]tar \[G] \[B7] \[Em] \\
		\endverse
		
		\beginverse
		\[E7]Bai\[Am]lé hasta el fi\[D7]nal... engualicha\[G]o, bailé hasta el amane\[B7]cer \\
		\[Em]Agitando pa\[D]ñuelos bai\[C]lé. ¡Qué lindo \[B7]bailar las zambas de \[Am]ayer! \[G] \[B7] \[Em] \\
		\endverse
		
		\endsong
		
 
\beginsong{Al Jardín De La República}[by={Virgilio Carmona}]

		\phantomsection  \addcontentsline{toc}{section}{Al Jardín De La República} 
 \label{sec:Al_Jardín_De_La_República} \beginverse
		\[G]Desde el \[D7]norte traigo en el \[G]alma \\
		\[D7]la alegre \[G]zamba que canto a\[D7]quí \\
		\[G]y que \[B7]bailan los tucu\[Em]manos \\
		\[B7]con entusiasmo \[Em]propio de \[D]allí, \\
		\[C]cada \[B7]cual sigue a su pa\[Em]reja, \\
		\[B7]joven o vieja, \[Em]de todo vi.
		\endverse
		
		\beginverse
		\[G]Media \[D7]vuelta y la compa\[G]ñera \\
		\[D7]forma una \[G]rueda para se\[D7]guir, \\
		\[G]viene el \[B7]gaucho, le hace un flo\[Em]reo \\
		\[B7]y un zapateo co\[Em]mienza \[D]allí, \\
		\[C]sigue el \[B7]gaucho con su flo\[Em]reo \\
		\[B7]y el zapateo ter\[Em]mina allí.
		\endverse
		
		\beginchorus
		\[Em]Para las \[E7]otras \[Am]no, \\
		\[C]Pa' las del \[B7]norte \[Am]sí, \\
		\[B7]para las Tu\[Em]cumanas, \\
		\[B7]mujer ga\[Em]lana, \[D]naranjo en \[C]flor \\
		\[B7]todo lo que ellas \[Em]quieran \\
		\[B7]que la primera ya \[Em]terminó.
		\endchorus
		
		\beginverse
		\[G]No me \[D7]olvido, viera, com\[G]padre, \\
		\[D7]de aquellos \[G]bailes que hacen a\[D7]llí \\
		\[G]tucuma\[B7]nos y tucuma\[Em]nas, \\
		\[B7]todos se afanan por \[Em]diver\[D]tir \\
		\[C]y hacer \[B7]linda esta triste \[Em]vida, \\
		\[B7]así se olvida que \[Em]hay que morir.
		\endverse
		
		\beginverse
		\[G]Empa\[D7]nadas y vino en \[G]jarra, \\
		\[D7]una gui\[G]tarra, bombo y vio\[D7]lín, \\
		\[G]y unas \[B7]cuantas mozas bi\[Em]zarras \\
		\[B7]pa' que la farra pueda \[Em]se\[D]guir \\
		\[C]sin que \[B7]falten esos cole\[Em]ros, \\
		\[B7]viejos cuenteros, \[Em]pa' que hagan reír.
		\endverse
		\endsong
 
\beginsong{De Mi Madre}[by={José Ignacio Rodríguez}]
    \phantomsection  \addcontentsline{toc}{section}{De Mi Madre} 
 \label{sec:De_Mi_Madre} \beginverse
    \[C]Volveré... \[G7]Volveré...
    Me espera la \[C]noche, vestida de \[G7]azul \\
    Y hasta el arro\[E7]yito que baja del \[Am]cerro,
    tra\[F]erá recu\[C]erdos de m\[G7]i ju\[C]ventud  \\
    \endverse

    \beginverse
    \[C]Volveré... \[G7]Volveré...
    Donde está mi \[C]madre esperándome\[G7]... \\
    De nuevo en sus [E7]brazos volver a ser \[Am]niño,
    viv\[F]ir como \[C]solo se v\[G7]ive una \[C]vez \\
    \endverse

    \beginchorus
    \[F]Azahar de blancos \[C]jazmines,
              \[G7]que aroman el patio del \[C]viejo jardín \\
           \[E7]Un beso de luna, me \[Am]espera en los valles,
    mi \[F]rancho, mi \[C]madre, todo mi \[G7]sentir \[C] \\
    \endchorus

    \beginverse
    \[C]Volveré... \[G7]Volveré...
    Por ese ca\[C]mino que ayer me ale\[G7]jó \\
    Al rumbo del \[E7]ave que vuelve a su \[Am]nido,
    busca\[F]ndo un \[C]alivio pa\[G7]ra su \[C]dolor  \\
    \endverse

    \beginverse
    \[C]Volveré... \[G7]Volveré...
    Y lejos la \[C]noche repite mi \[G7]voz \\
    La voz de un \[E7]cariño que lejos se \[Am]siente,
    y ll\[F]ama al \[C]ausente de su co\[G7]razón \[C] \\
    \endverse

\endsong
 
\beginsong{La Pomeña}[by={Cuchi Leguizamón}]
		\transpose{-3}

		\phantomsection  \addcontentsline{toc}{section}{La Pomeña} 
 \label{sec:La_Pomeña} \beginverse
		\[G7M]Eulogia Tapia en \[Em]La Poma \[Am6]
		Al \[Am7]aire \[D7]da su ter\[G7M]nura \\
		\[Dm7]Si pasa sobre la \[Db7]arena \[C7M/G]
		Iba pi\[Am6]sando la \[D7]luna \[G7M] \\
		\[F7]Si pasa sobre la \[E7]arena \[A6]
		Iba pi\[Am6]sando la luna \[G#7M] \[G7M]
		\endverse
		
		\beginverse
		\[G7M]El trigo que va cor\[Em]tando \[Am6]
		Ma\[Am7]dura \[D7]por su cin\[G7M]tura \\
		\[Dm7]Mirando flores de \[Db7]alfalfa \[C7M/G]
		Sus ojos \[Am6]negros se a\[D7]zulan \[G7M] \\
		\endverse
		
		\beginchorus
		\[G]El sauc\[G/F#]e de tu ca\[F7]sa 
		Te está llo\[E]rando \[Am6] \\
		\[Cm7]Por qué te roban Eu\[F7]logia \[Bm7]
		Carnava\[D7sus4]leando \[G#7M] \[G7M] \\
		\[Bm]Por que te roban Eu\[E7(9)]logia \[Am6]
		Carnava\[D7]leando \[G#maj7] \[Gmaj7]
		\endchorus
		
		\beginverse
		\[G7M]La cara se le enha\[Em]rina \[Am6]
		La \[Am7]sombra \[D7]se le ena\[G7M]rena \\
		\[Dm7]Cantando y desenca\[Db7]ntando \[C7M/G]
		Se le entre\[Am6]veran las \[G#7M]penas \[G7M]
		\endverse
		
		\beginverse
		\[G7M]Viene en un caballo \[Em]blanco \[Am6]
		La \[Am7]caja \[D7]en sus manos \[G7M]tiembla \\
		\[Dm7]Y cuando se hunde en la \[Db7]noche \[C7M/G]
		Es una \[Am6]dalia mo\[G#7M]rena \[G7M]
		\endverse
		\endsong
 
\beginsong{La Tempranera}[by={Eduardo Falú}]
			
			\phantomsection  \addcontentsline{toc}{section}{La Tempranera} 
 \label{sec:La_Tempranera} \beginverse
			\[Em]Eras la temprane\[B7]ra \\
			niña pri\[E7]mera, amane\[Am]cida flor, \\
			suave rosa ga\[Em]lana \\
			la más bo\[F#7]nita Tucuma\[B7]na \[E7] \\
			\[Am]suave rosa ga\[Em]lana \\
			la más bo\[B7]nita Tucuma\[Em]na.
			\endverse
			
			\beginverse
			Frente de adoles\[B7]cente \\
			gentil mi\[E7]lagro de tu tri\[Am]gueña piel, \\
			negros ojos sin\[Em]ceros \\
			paloma \[F#7]tibia de Monte\[B7]ros \[E7] \\
			negros ojos sin\[Am]ceros \\
			paloma \[B7]tibia de Monte\[Em]ros.
			\endverse
			
			\beginchorus
			\[E7]Al bailar esta \[Am]zamba fue \\
			\[D7]que rendido te \[G]amé \[B7] \[E7] \\
			\[Am]Eras la temprane\[Em]ra \\
			de mis ar\[F#7]restos prisione\[B7]ra \[E7] \\
			\[Am]Mía... ya te sa\[Em]bía \\
			cuando por \[B7]fin te coro\[Em]né.
			\endchorus
			
			\beginverse
			Eras la prime\[B7]vera \\
			la pregone\[E7]ra del delic\[Am]ado amor, \\
			lloro amarga\[Em]mente \\
			aquel ro\[F#7]mance adolescen\[B7]te \[E7] \\
			lloro amarga\[Am]mente \\
			aquel ro\[B7]mance adolescen\[Em]te.
			\endverse
			\endsong
			
			
			
 
\beginsong{La Viajerita}[by={Atahualpa Yupanqui}]
			\phantomsection  \addcontentsline{toc}{section}{La Viajerita} 
 \label{sec:La_Viajerita} \beginverse
			\[Am]Desde los cerros \\
			\[Am]baja esta \[E7]zámbita, \\
			\[Am]por eso la llamo yo \\
			\[F]la viajerita, \[C]palomi\[E7]tay. \[Am]
			\endverse
			
			\beginverse
			\[Am]Sendas de \[E7]arena, \\
			\[Am]arcos flo\[E7]ridos, \\
			\[F]y un corazón que pena \\
			\[C]por un oli\[E7]vido, \[Am]palomi\[E7]tay. \[Am]
			\endverse
			
			\beginchorus
			\[G7]¡Ay, viajeri\[C]ta! \\
			\[G7]el alba a\[C]soma, \\
			\[F]trayendo de los cerros, \\
			\[C]frescor y a\[E7]roma, \[Am]palomi\[E7]tay. \[Am]
			\endchorus
			
			\beginverse
			\[Am]Yo soy de a\[E7]rriba, \\
			\[Am]soy de Go\[E7]chula, \\
			\[F]ranchitos, montes, ríos, \\
			\[C]soles y lu\[E7]nas, \[Am]palomi\[E7]tay. \[Am]
			\endverse
			
			\beginverse
			\[Am]Hasta el Pachi\[E7]vi, \\
			\[Am]voy los do\[E7]mingos, \\
			\[F]y por la noche al cerro, \\
			\[C]vuelvo soli\[E7]to, \[Am]palomi\[E7]tay. \[Am]
			\endverse
			\endsong
 
\beginsong{Luna Tucumana}[by={Atahualpa Yupanqui}]
		\phantomsection  \addcontentsline{toc}{section}{Luna Tucumana} 
 \label{sec:Luna_Tucumana} \beginverse
		\[E7]Yo no le canto a la \[Am]luna
		Porque \[E7]alumbra y nada \[Am]más \[A7] \\
		\[Dm]Le canto porque ella \[Am]sabe
		De mi largo \[E7]caminar \[Am] \\
		\endverse
		
		\beginverse
		\[E7]Hay lunita tucuma\[Am]na
		Tamborcito \[E7]calcha\[Am]quí \[A7] \\
		\[Dm]Compañera de los \[Am]gauchos
		En las sendas de \[E7]Tafí \[Am] \\
		\endverse
		
		\beginchorus
		\[G]Perdido en las cerra\[C]zones
		Quien sabe \[G7]vidita por \[F7]donde \[C]andaré \[A7] \\
		\[Dm]Mas cuando salga la \[Am]luna
		Canta\[E7]ré, canta\[Am]ré \[A7] \\
		\[Dm]A mi Tucumán que\[Am]rido
		Canta\[E7]ré, canta\[Am]ré, canta\[E7]ré \[Am]
		\endchorus
		
		\beginverse
		\[E7]Con esperanza o con \[Am]pena
		En los campos de \[E7]acheral \[Am] \[A7] \\
		\[Dm]Yo he visto a la luna \[Am]buena
		Besando el caña\[E7]veral \[Am] \\
		\endverse
		
		\beginverse
		\[E7]Si en algo nos parece\[Am]mos
		Luna de la so\[E7]ledad \[Am] \[A7] \\
		\[Dm]Yo voy andando y can\[Am]tando
		Que es mi modo de \[E7]alumbrar \[Am] \\
		\endverse
		
		\endsong
 
\beginsong{Saltita}[by={Roberto Ternán}]
		\phantomsection  \addcontentsline{toc}{section}{Saltita} 
 \label{sec:Saltita} \beginverse
		\[Am]Me sigue tu re\[F]cuerdo \[G7]por donde \[C]vaya \\
		\[D]Y en la zamba carpera m\[F]e llora el a\[G7]lm\[C]a \\
		\[E7]Y en la zamba car\[Am]pera, Saltita,  \[Dm]me ll\[C]ora \[E7]el al\[Am]ma.
		\endverse
		
		\beginverse
		Andando solo y \[F]lejos, \[G7]tierra de \[C]Güemes \\
		\[D]Lágrimas de nostalgias, br\[F]otan a\[G7]vec\[C]es \\
		\[E7]Lágrimas de nos\[Am]talgias, Saltita, b\[Dm]rot\[C]an a\[E7] ve\[Am]ces.
		\endverse
		
		\beginchorus
		\[F]Pura nostalgias teng\[C]o, \\
		\[D]cuando me acuerdo de \[C]Salta \\
		\[D]Por eso coca y vica nun\[F]ca me f\[G7]alt\[C]a \\
		\[E7]Soñando carnav\[Am]ales, Saltita, s\[Dm]e v\[C]uelve \[E7]mi al\[Am]ma.
		\endchorus
		
		\beginverse
		Desde la cruz del \[F]cerro \[G7]ya se adi\[C]vina \\
		\[D]que por tus calles corren, ca\[F]nto y p\[G7]oes\[C]ía \\
		\[E7]que por tus calles \[Am]corren, Saltita, c\[Dm]an\[C]to y p\[E7]oes\[Am]ía.
		\endverse
		
		\beginverse
		Digo tu nombre y \[F]suena \[G7]chura la \[C]zamba \\
		\[D]como para irse yendo d\[F]e cac\[G7]har\[C]pallas \\
		\[E7]como para irse \[Am]yendo, Saltita, d\[Dm]e c\[C]ach\[E7]arpal\[Am]las.
		\endverse
		\endsong
 
\beginsong{Zamba Del Laurel}[by={Cuchi Leguizamón}]
		\phantomsection  \addcontentsline{toc}{section}{Zamba Del Laurel} 
 \label{sec:Zamba_Del_Laurel} \beginverse
		\[G]Si lo verde tu\[E7]viera otro \[Am]nombre
		\[D7]Debería llamarse ro\[G]cío \\
		\[Bm7(b5)]Si pudiera vol\[E7]ver desde el \[Am]agua al laurel
		\[D7]Volvería a la in\[G]fancia del río
		\endverse
		
		\beginverse
		\[G]En lo verde lau\[E7]rel de tus \[Am]ojos
		\[D7]El misterio del bosque se a\[G]soma \\
		\[Bm7(b5)]Y la vida otra \[E7]vez vuelve \[Am]flor de tu piel
		\[D7]Bajo un sol de mu\[G]chacha y aroma
		\endverse
		
		\beginchorus
		\[C#m7(b5)]Déjame en lo \[F#7]verde celebrar el \[Bm7]día
		Porque por lo \[Am]verde regreso a la \[B7]vida \[Em] \\
		\[Cmaj7]Yo \[C#dim]muero para \[Bm7]volver
		Juntando ro\[Am]cío en la \[D7]flor del lau\[Dm]rel \[G7] \\
		\[Cmaj7]Yo \[C#dim]muero para \[G]volver
		Juntando ro\[Am]cío en la \[D7]flor del lau\[G]rel
		\endchorus
		
		\beginverse
		\[G]Si lo verde su\[E7]piera tu \[Am]nombre
		\[D7]La ternura no me olvida\[G]ría \\
		\[Bm7(b5)]Porque viene de \[E7]vos, puro y \[Am]simple el verdor
		\[D7]Como el simple ver\[G]dor de la vida
		\endverse
		
		\beginverse
		\[G]Se me ha vuelto co\[E7]gollo el si\[Am]lencio
		\[D7]De esperarte a la orilla del \[G]río \\
		\[Bm7(b5)]Y me gusta sa\[E7]ber que un a\[Am]roma a laurel
		\[D7]Te llenó de ro\[G]cío el olvido
		\endverse
		\endsong
 
\beginsong{Zamba Por Vos}[by={Alfredo Zitarrosa}]
    \phantomsection  \addcontentsline{toc}{section}{Zamba Por Vos} 
 \label{sec:Zamba_Por_Vos} \beginverse
             
    \[C]Yo no canto por vos \[G7]\\
    te canta la zamba \[C]\\
    y dice al cantar\[G7] \\
    no te pu\[F]edo olv\[C]idar \\
    no te pu\[G7]edo olvid\[C]ar.
    \endverse

    \beginverse
    \[C]Yo no canto por vos \[G7]\\
    te canta la zamba \[C]\\
    y cantando así\[G7] \\
    canta p\[F]ara m\[C]í \\
    canta pa\[G7]ra mí\[C].
    \endverse

    \beginchorus
    \[C7]Zambita canta\[F] \\
    no la esperes más \[C] \\
    tenés qu\[D]e pens\[G7]ar \\
    que si \[F]no volv\[C]ió \\
    es que y\[G7]a te olv\[C]idó \\
    perfum\[C]a esa fl\[G7]or \\
    que se ma\[F]rchi\[C]tó \\
    que se mar\[G7]chi\[C]tó.
    \endchorus

    \beginverse
    \[C]Yo tuve un amor \[G7]\\
    lo dejé esperando\[C] \\
    y cuando volv\[G7]í \\
    no l\[F]a conocí\[C] \\
    no l\[G7]a conocí.\[C]
    \endverse

    \beginverse
    \[C]Dijo que tal vez \[G7]\\
    me estuviera amando\[C] \\
    me miró y se fu\[G7]e \\
    sin d\[F]ecir por qué\[C] \\
    sin dec\[G7]ir por qu\[C]é.
    \endverse

 
\endsong
 
